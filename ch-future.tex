
In this work I derived and implemented a two grid acceleration scheme for the nonlinear diffusion acceleration equations with neutron upscattering. In the test problems, a reduction in the number of Gauss-Seidel iterations necessary to resolve the upscattering is observed with roughly a factor of three improvement with no noticeable loss to accuracy. 

The current implementation is a proof of concept code developed in python. The Slaybaugh Lab intends to implement a performance optimized C++ version into the our neutron transport code, Bay Area Radiation Transport. Such an implementation would enable testing larger problems and aid in a better understanding of how the method scales.

\section{Possible Extensions to TG-NDA}
There are a number of modifications to my implementation of TG-NDA that could be interesting to explore. 


\subsection{Applications to Criticality Problems}
In this work I only focused on fixed-source problems for shielding applications, however many materials commonly used in nuclear reactors, such as water, heavy water, or graphite, also exhibit significant upscattering. Performing criticality calculations using our method would be a straightforward extension involving only a layer of eigenvalue iteration to wrap around the existing implementation. 


\subsection{Compatibility with Other Methods}
In my implementation, I made choices regarding the discretization schemes, iterative methods, and equations I chose to use based on what I felt to be in common usage in the community. However, there are several other options to explore. In particular, 
\begin{enumerate}
\item $P_N$ Angular Discretization:
Our derivation is specific to the $S_N$ equations. It is possible to derive the higher order equation using $P_N$ instead \cite{zheng-thesis} and see if there is any effect on the convergence behavior of TG-NDA. 
\item Discontinuous Finite Elements:
In my implementation I use continuous finite elements to discretize in space. Discretizing NDA using discontinuous finite elements, as done in \cite{Schunert2017}, would change the form of TG-NDA and could change Gauss-Seidel convergence behavior. 
\item Other Higher Order Equations:
I chose to pair NDA with SAAF, however other equations could be used as well, such as the even-parity equation \cite{Noh1996}.
\item Other Multigroup Solvers:
TG-NDA is specific to Gauss-Seidel, but a similar procedure could be used to create an acceleration scheme for other multigroup solvers such as Point Jacobi.
\end{enumerate}

