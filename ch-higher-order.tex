In calculating the drift vector (Eq. 3), we use the scalar flux calculated by a ``higher-order equation." In this case, we will be using the Self-Adjoint Angular Flux Equation (SAAF) \cite{saaf} as our higher order equation. 

\section{Fixed Source Formulation}
The monoenergetic version of the Self-Adjoint Angular Flux equation appropriate for $S_n$ calculations is as follows:
\begin{equation}
    - \vec{\Omega} \cdot \vec{\nabla}\frac{1}{\sigma_t}\vec{\Omega} \cdot \vec{\nabla} \psi + \sigma_t \psi = S\psi + q - \vec{\Omega} \cdot \vec{\nabla} \frac{(S\psi + q)}{\sigma_t}
    \label{eq:SAAF}
\end{equation}
Where $\vec{\Omega}$ are the angles, $\sigma_t$ is the total cross section, $\psi$ is the angular flux, $S\psi$ is the scattering source, and $q$ is the fixed source (for ease, we write $S\psi + q$ as $Q$ to represent the total source.

\section{$k$-eigenvalue Formulation}
To find the criticality state of a reactor, we must find $k$, the ratio of neutrons in two successive generations.  This requires turning the steady state form of SAAF into an eigenvalue problem by replacing the external fixed source, $q$ with a fission source, $\frac{1}{k}\chi\nu \sigma_f$. This gives the following equation,
\begin{equation}
        - \vec{\Omega} \cdot \vec{\nabla}\frac{1}{\sigma_t}\vec{\Omega} \cdot \vec{\nabla} \psi + \sigma_t \psi = S\Psi + \frac{1}{k}\frac{\chi}{4\pi}\nu\sigma_f\phi - \vec{\Omega} \cdot \vec{\nabla} \frac{(S\Psi + \frac{1}{k}\frac{\chi}{4\pi}\nu\sigma_f\phi)}{\sigma_t}
    \label{eq:SAAF-eigenvalue}
\end{equation}