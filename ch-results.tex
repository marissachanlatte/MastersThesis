\section{One Material}
% Did you tell us how the TG method works in the method section?
%
For the first test problem, we used a single material with upscattering throughout a 20 by 20
% what unit??
 domain with a constant source. 
% of what energy? At what location? 
 We used cross sections for the seven-group moderator material in the C5G7 benchmark problem. \cite{C5G7}.  
The problems were run on a single processor of a MacBook Pro and the results are show in Table~\ref{tab:onemat}.
% what are the details of the processor?
\begin{table}[!htb]
\centering
\caption{Runtime and GS Iteration Count for One Material Problem}
    \label{tab:onemat}
\begin{center}
    \begin{tabular}{|c|c|c|}
    \hline
    & Runtime (s) & GS Iterations \\
    \hline
    TG-NDA & 5,465 & 9 \\
    NDA & 14,513 & 31 \\
    \hline
    \end{tabular}
\end{center}
\end{table}
% match # of sigfigs... also, I just cut the fractions of seconds as they seemed unimportant  and added commas for readability. Could also covert to sci notation.
The two-grid method provides a considerable acceleration of the Gauss Seidel method. While when using TG-NDA each iteration takes slightly longer as the correction term must be calculated, there is a considerable decrease in the number of iterations necessary to reach convergence. 
% may want to add average time / iteration to table to facilitate the comparison

Importantly, the acceleration in convergence came at no cost to accuracy. As can be seen in Figure~\ref{fig:Moderator}, the NDA and TG-NDA solutions have the same values (up to tolerance). In the interest of space, we only show the results from the highest energy group, but the agreement between NDA and TG-NDA held for all energy groups in our test problems. 
% and the full set of data can be found at location...
\begin{figure}[H]
\centering
\begin{subfigure}{.5\textwidth}
  \centering
  \includegraphics[width=\linewidth]{fig/nda_c5g7mod_scalar_flux_group0.png}
  \caption{Highest Energy Group, NDA Scalar Flux}
  \label{fig:NDA-Mod}
\end{subfigure}%
\begin{subfigure}{.5\textwidth}
  \centering
  \includegraphics[width=\linewidth]{fig/tgnda_c5g7mod_scalar_flux_group0.png}
  \caption{Highest Energy Group, TG-NDA Scalar Flux}
  \label{fig:TG-NDA-Mod}
\end{subfigure}
\caption{Comparison of NDA and TG-NDA in Scalar Flux for the One Material Problem}
\label{fig:Moderator}
\end{figure}
% please add units

\section{Two Materials}
The second problem consists of two materials in a concentric, box geometry, shown in Figure ~\ref{fig:test_geometry}. 
The first material, located in the center and outer layer, is the C5G7 moderator material used above and the second material has the same total cross sections and pattern of upscattering, but with higher absorption and lower total scattering. Both materials have seven groups. 
% can you include a table of the cross section values used? At least for the scattering? It would help us see the structure since this depends a lot on the scattering structure.  
There is a box source in the center that emits 70\% in the highest energy group, 20\% in the second-highest, and 10\% in the third-highest energy group.
\begin{figure}[H]
    \centering
    \includegraphics[width=.3\textwidth]{fig/Geometry.png}
    \caption{Geometry of Two-Material Test Problem}
    \label{fig:test_geometry}
\end{figure}

\begin{table}[!htb]
\centering
\caption{Runtime and GS Iteration Count for Two Material Problem}
    \label{tab:two}
\begin{center}
    \begin{tabular}{|c|c|c|}
    \hline
    & Runtime (s) & GS Iterations \\
    \hline
    TG-NDA & 4,222 & 8 \\
    NDA & 10,382 & 25 \\
    \hline
    \end{tabular}
\end{center}
\end{table}

The test results are in Table ~\ref{tab:two}. Again, TG-NDA showed a significant improvement over the unaccelerated Gauss Seidel, taking roughly one third the number of iterations as NDA alone. Again, NDA and TG-NDA agree in terms of flux values. Figure~\ref{fig:Moderator2} also only shows results for the highest energy group 
% and again, the full set of data can be found... 
\begin{figure}[H]
\centering
\begin{subfigure}{.5\textwidth}
  \centering
  \includegraphics[width=\linewidth]{fig/nda_iron-water_scalar_flux_group0.png}
  \caption{Highest Energy Group, NDA}
  \label{fig:NDA-Mod2}
\end{subfigure}%
\begin{subfigure}{.5\textwidth}
  \centering
  \includegraphics[width=\linewidth]{fig/tgnda_iron-water_scalar_flux_group0.png}
  \caption{Highest Energy Group, TG-NDA}
  \label{fig:TG-NDA-Mod2}
\end{subfigure}
\caption{Comparison of NDA and TG-NDA in Scalar Flux for the Two Material Problem}
\label{fig:Moderator2}
\end{figure}

\section{Reproducibility}
The code used to run these experiments is hosted online at www.github.com/mzweig/gallo. The version used is tagged as masters-thesis. The geometry inputs \texttt{origin-centered10} and material input \texttt{c5g7mod} were used for the first test problem, and the geometry input \texttt{iron-water10} and material input \texttt{mod-water} were used for the second problem. 