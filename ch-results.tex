\section{One Material}
For the first test problem we used a single material with significant upscattering throughout the domain with a constant source. 
\subsection{Cross Sections}
For experimentation purposes, we created a fake seven group material with significant upscattering. The scattering block is given below.
\begin{table}
\begin{center}
    \begin{tabular}{|c|c|c|c|c|c|c|c|}
    \hline
    & g'=1 & g'=2 & g'=3 & g'=4 & g'=5 & g'=6 & g'=7  \\
    \hline    
    g = 1 & 10 & 1 & 1 & 1 & 0 & 0 & 0  \\
    \hline    
    g = 2 & 0 & 10 & 1 & 1 & 1 & 0 & 0  \\
    \hline    
    g = 3 & 0 & 0 & 10 & 1 & 1 & 1 & 0  \\
    \hline    
    g = 4 & 0 & 0 & 0 & 10 & 1 & 1 & 1  \\
    \hline    
    g = 5 & 0 & 0 & 0 & 1 & 10 & 1 & 1  \\
    \hline    
    g = 6 & 0 & 0 & 0 & 1 & 1 & 10 & 1  \\
    \hline    
    g = 7 & 0 & 0 & 0 & 1 & 1 & 1 & 10  \\
    \hline
    \end{tabular}
\end{center}
\caption{Scattering Matrix for Test Material}
\end{table}
\subsection{Timing Results}
We ran the problems three times on a single processor of a MacBook Pro. 
\begin{center}
    \begin{tabular}{|c|c|c|}
    \hline
    & Runtime (s) & GS Iterations \\
    \hline
    TG-NDA & 2850.48293 & 7 \\
    NDA & 4352.18865 & 11 \\
    \hline
    \end{tabular}
\end{center}
\section{Two Materials}
The second problem consists of two materials in a concentric geometry with a box source in the center. 
\begin{figure}
    \centering
    \includegraphics[width=.3\textwidth]{fig/Geometry.png}
    \caption{Geometry of Two-Material Test Problem}
    \label{fig:test_geometry}
\end{figure}