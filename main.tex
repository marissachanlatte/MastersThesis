\documentclass[12pt]{report}
\usepackage[utf8]{inputenc}
\usepackage{amsmath}
\usepackage{graphicx}
\usepackage{subcaption}
\usepackage{caption}
\usepackage{tabls}
\usepackage{afterpage}
\usepackage{epsf}
\usepackage{color}
\usepackage{algorithm}
\usepackage[noend]{algpseudocode}
\usepackage{lmodern}
\usepackage{lipsum}
\usepackage{todonotes}
\usepackage{wrapfig}
\usepackage[titletoc]{appendix}
% necessary commands
\newcommand{\chig}{\chi_\rg}
\newcommand{\sn}{S$_N$}
\newcommand{\varphigp}{\varphi_{\rg'}}
\newcommand{\varphig}{\varphi_\rg}
\newcommand{\need}[1]{\textcolor{red}{#1}}
\newcommand{\md}{\mathcal{D}}
\newcommand{\pd}{\partial\md}
\newcommand{\mm}[1]{\mathrm{#1}}
\newcommand{\rg}{\mm{g}}
\newcommand{\mrm}{\mm{m}}
\newcommand{\rgp}{{\rg'}}

\usepackage{geometry}
\geometry{margin=1in}

\title{A Two-Grid, Nonlinear Diffusion Acceleration Method for the $S_N$ Equations with Neutron Upscattering}
\author{Marissa Ramirez Zweiger }
\date{July 27th, 2018}


\begin{document}
\pagestyle{empty}
\begin{center}
A Two-Grid, Nonlinear Diffusion Acceleration Method  \\ 
for the $S_N$ Equations with Neutron Upscattering \\
\vspace{20mm}
by \\
\vspace{5mm}
Marissa Ramirez Zweiger \\ 
\vspace{20mm}
A thesis submitted in partial
satisfaction of the \\
\vspace{5mm}
requirements for the degree of \\
\vspace{5mm}
Master of Science \\
\vspace{5mm}
in \\
\vspace{5mm}
Nuclear Engineering \\
\vspace{5mm}
in the \\
\vspace{5mm}
Graduate Division \\
\vspace{5mm}
of the \\
\vspace{5mm}
University of California, Berkeley \\
\vspace{20mm}
Committee in charge: \\
\vspace{5mm}
Professor Rachel Slaybaugh, Chair \\
Professor Phillip Colella \\
Professor Massimiliano Fratoni \\
\vspace{5mm}
Summer 2018
\end{center}




\renewcommand{\abstractname}{Acknowledgements}
\begin{abstract}
\pagenumbering{roman}
\thispagestyle{plain}
I would like to acknowledge my advisers, Dr. Rachel Slaybaugh and Dr. Phillip Colella, as well as my third committee member, Dr. Max Fratoni, for their guidance. This work would not have been possible without the mentorship of Dr. Weixiong Zheng - Thank you. I am also particularly appreciative of my colleagues, Joshua Rehak and Mario Ortega who were always willing to talk through concepts and let me bounce ideas off of them. 

Most of all, my gratitude goes to my family. Thank you to my mother, Andrea Ramirez, who has always been a constant source of support and to my partner in all things, Marcell Vazquez-Chanlatte, for always having handy the things I've needed most: useful python tricks, a blackboard, and a warm cup of tea. 
\end{abstract}
\setcounter{page}{2}
\pagestyle{plain}
\tableofcontents

\renewcommand{\abstractname}{Abstract}
\begin{abstract}
\pagenumbering{arabic}
\thispagestyle{plain}
Nonlinear diffusion acceleration (NDA), also known as coarse-mesh finite difference, is a well-known technique applied to accelerate the scattering convergence in neutronics calculations. In multigroup problems, NDA is effective in conjunction with Gauss-Seidel iteration in energy when there is not much upscattering. However, in the presence of significant upscattering, particularly when there is little leakage or absorption, which is common in thermal reactor problems, the efficiency of Gauss-Seidel with NDA degrades.
For use with other transport formulations, a two-grid (TG) acceleration scheme was developed to remedy the issue in Gauss Seidel with upscattering. It approximates iteration error by solving a one-group diffusion-like equation with artificial material properties generated by using the scattering eigen-spectrum. Inspired by this work, we derive a two-grid scheme specific for the NDA equation with Gauss-Seidel iteration.
\end{abstract}

\chapter{Introduction}
\setcounter{page}{2}
\pagestyle{plain}
\pagenumbering{arabic}
\label{sec:intro}
Nonlinear diffusion acceleration (NDA), also known as coarse-mesh finite difference, is a well-known technique applied to accelerate the scattering convergence in neutronics calculations. In multigroup neutronics problems, NDA is effective in conjunction with Gauss-Seidel (GS) iteration in energy if there is little upscattering. However, when upscattering manifests, which is common with thermal neutrons, the efficiency of GS-NDA degrades as extra iterations are required for convergence in energy \cite{park-nda}.

To remedy the issue in GS with upscattering, an energy two-grid (TG) acceleration scheme was first developed to approximate iteration error by solving a one-group diffusion-like equation with artificial material properties generated by using the scattering eigen-spectrum \cite{morel-upscat}. Later, a transport TG (TTG) method was developed that approximates the energy error using a consistent \sn\ solver in multi-D \cite{evans-upscat}. Inspired by the previous studies, we derive a TG scheme for the NDA equation with GS iteration.

\section{Background}
\subsection{Motivation}
To accelerate the convergence of source iteration in transport calculations, Nonlinear Diffusion Acceleration (NDA) was developed. It pairs a lower order equation with a higher order, drift diffusion equation. As the higher order equation is not conservative, formally the method is inconsistent: The scalar flux and current may not be equal upon convergence, although they are in the limit as the spatial mesh is refined. However, second order accuracy can still be maintained with the high-order equation generally giving a more accurate shape and the conservative low-order equation giving a more accurate magnitude. The combination of the two equations has been found to be more accurate than either equation alone. \cite{morel-holo} \par
The steady-state transport equation is dependent on space, angle, and energy. It is often solved via a series of nested iterations. The various iteration methods and how they are used with each other are described in detail in Ch. \ref{sec:iterative}. The Nonlinear Diffusion Acceleration happens in the source iteration, where angle and energy are fixed. When energy is discretized into more than one energy group, an outer layer of iteration is introduced. One of the most commonly used methods for iterating over energy groups is known as Gauss-Seidel. Gauss-Siedel is guaranteed to converge, however in problems with significant upscattering the time it takes to reach convergence can become arbitrarily slow. 
\par
A number of techniques to accelerate Gauss-Seidel convergence have been developed, although to our knowledge none have yet been paired with NDA.The primary techniques used in commercial transport software rely on a rebalance scheme and coarse-mesh finite-difference diffusion. Although these methods are widely used and successful for acceleration, they are very sensitive to the coarse mesh size. Rebalancing with too fine a mesh may be divergent, and an overly coarse mesh degrades performance. \cite{evans-upscat}
\par
While for standard problems, the proper mesh size is generally well understood, this is not the case for shielding problems. Adams and Morel developed an upscatter acceleration scheme known as the Two-Grid method. An estimation of the error at each Gauss-Seidel iteration is calculated using a collapsed in energy, one-group diffusion equation and energy eigenvector of the Gauss-Seidel iteration matrix. These make up the correction term to the scalar flux at each group which is applied at each iteration. In one dimensional calculations, Adams and Morel have demonstrated their method to be very efficient for thermal upscattering problems. \cite{morel-upscat}

\section{The Boltzmann Transport Equation}
The angular neutron flux of a reactor, $\psi$ can be described by the steady state Boltzmann Transport equation.

\begin{equation}
\begin{split}
 [\hat{\Omega} \cdot \nabla + \Sigma(\vec{r}, E)]\psi(\vec{r}, \hat{\Omega}, E) &= \chi(E) \int_0^\infty dE' \nu \Sigma_{f}(\vec{r}, E') \int_{4\pi} d\hat{\Omega}'\psi(\vec{r}, \hat{\Omega}, E') \\   &+ \int_0^\infty dE' \int_{4\pi} d\hat{\Omega}' \Sigma_s(\vec{r}, E' \rightarrow E, \hat{\Omega}' \cdot \hat{\Omega})\psi(\vec{r}, \hat{\Omega}', E')   
\end{split}
\label{eq:transport}
\end{equation}


where $\hat{\Omega}$ represents the angle; $\vec{r}$, the position vector; $E$, the energy; $\Sigma$, the total macroscopic cross-section; $\Sigma_f$, the macroscopic fission cross-section; $\Sigma_s$, the macroscopic scattering cross section; $\chi$, the energy distribution; and $\nu$, the average number of neutrons per fission. 

\subsection{Forms of the Transport Equation}
There are are several forms of the transport equaution that are of interest in the field of nuclear energy. In this work we present all methods in fixed-source form, however they can all be easily extended to $k$-eigenvalue form for criticality calculations. 

\subsubsection{Fixed Source Form}
In the fixed source form, we assume there is no fission and that there is an external neutron source, $Q$.

\begin{equation}
\begin{split}
 [\hat{\Omega} \cdot \nabla + \Sigma(\vec{r}, E)]\psi(\vec{r}, \hat{\Omega}, E) &= \\ \int_0^\infty dE' &\int_{4\pi} d\hat{\Omega}' \Sigma_s(\vec{r}, E' \rightarrow E, \hat{\Omega}' \cdot \hat{\Omega})\psi(\vec{r}, \hat{\Omega}', E')  +Q 
\end{split}
 \label{eq:transport_fixed_source}
\end{equation}

\subsubsection{$k$-Eigenvalue Form}
If the chain reaction is self-sustaining and time-independent, the reactor is known as ``critical." To maintain criticality, the asymptotic neutron distribution must not be changing over time. This behavior can be described by a parameter,  $k$, the ratio of neutrons in two successive generations. We scale $\nu$ in Eq. \eqref{eq:transport} by $k$ to express the deviation from critical. This gives the following equation,

\begin{equation}
    \label{eq:transport_eigenvalue}
    \begin{split}
        [\hat{\Omega} \cdot \nabla + \Sigma(\vec{r}, E)]\psi(\vec{r}, \hat{\Omega}, E) &= \frac{\chi(E)}{k} \int_0^\infty dE' \nu \Sigma_{f}(\vec{r}, E') \int_{4\pi} d\hat{\Omega}'\psi(\vec{r}, \hat{\Omega}, E') \\ &+ \int_0^\infty dE' \int_{4\pi} d\hat{\Omega}' \Sigma_s(\vec{r}, E' \rightarrow E, \hat{\Omega}' \cdot \hat{\Omega})\psi(\vec{r}, \hat{\Omega}', E') 
    \end{split}
\end{equation}

\eqref{eq:transport_eigenvalue} is simply an eigenvalue problem that with some algebraic manipulation can be thought of in the form $A\psi = \frac{1}{k} \psi$ and solved via standard eigenvalue solvers. 

\section{Useful Approximations}
In this work we are primarily concerned with the fixed source form of the transport equation, although all of our methods can be used with the eigenvalue form by replacing the fixed source $Q$ with $\frac{1}{k}\nu\Sigma_f$. For ease we assume our scattering and fixed sources are isotropic, which gives the following form

\begin{equation}
\begin{split}
 [\hat{\Omega} \cdot \nabla + \Sigma(\vec{r}, E)]\psi(\vec{r}, \hat{\Omega}, E) &= \\  \int_0^\infty \frac{1}{4\pi} &\Sigma_s(\vec{r}, E' \rightarrow E)  dE' \int_{4\pi} d\hat{\Omega}'\psi(\vec{r}, \hat{\Omega}', E')  + \frac{1}{4\pi}Q 
\end{split}
 \label{eq:transport_isotropic_scattering}
\end{equation}


This equation gives the angular flux. To find the scalar flux, $\phi$, we must integrate over all directions.
\begin{equation}
    \phi(\vec{r}, E) = \int_{4\pi} \phi(\vec{r}, \hat{\Omega}, E) d \Omega.
\end{equation}

\subsection{The Diffusion Equation}
To simplify even further, we employ a commonly used approximation known as the diffusion equation. To derive, assume all neutrons have the same energy and consider the neutron balance within an infinitesimal volume centered at a point, $r$. Under steady state conditions, neutron conservation requires

\begin{equation}
    \textit{neutrons leaking out} + \textit{neutrons absorbed} = \textit{source neutrons emitted}.
\end{equation}
We describe the neutrons leaking out as the rate of the current, $J$, in all directions, the neutrons absorbed is the absorption cross section times the scalar flux, $\Sigma_a\phi$, and the source neutrons are represented by the source variable, $Q$. 
\begin{equation}
    \vec{\nabla}\cdot \vec{J}(\vec{r}) + \Sigma_a(\vec{r})\phi(\vec{r}) = Q(\vec{r})
\end{equation}
Using Fick's Law which relates the current to the flux, $\vec{J}(\vec{r}) = -D(\vec{r})\vec{\nabla}\phi(\vec{r})$ where $D = 1/3\Sigma_t$, we get the diffusion approximation

\begin{equation}
\begin{split}
 - \nabla \cdot D(\vec{r})\nabla\phi(\vec{r}) &+ \Sigma_a \phi(\vec{r}) = Q(\vec{r}).
\end{split}
\label{eq:diffusion_fixed_source}
\end{equation}

While the diffusion equation is much easier to solve, due to the assumptions made in the Fick's Law approximation, it is not valid near boundaries where material properties change dramatically, near localized sources, or in strongly absorbing media \cite{lewis-miller}.

\subsection{Nonlinear Diffusion Accelration}
This work presents an acceleration to a method known as Nonlinear Diffusion Acceleration (NDA). NDA reformulates the transport equation as a correction to the diffusion equation and uses a two step process to solve. For reference, we repeat the derivation of the low-order NDA equation found in \cite{morel-holo} with small modifications assuming no fission source and vacuum boundary conditions. Consider the first order, one-group, fixed-source, steady-state \sn transport equation with isotropic scattering. 

  \begin{equation}
  \vec{\Omega}\cdot \vec{\nabla} \psi \left(\vec{r}\right)+ \Sigma_{\mm{t}}\left(\vec{r}\right)\psi = \frac{1}{4 \pi} \Sigma_{\mm{s}}\left(\vec{r}\right) \phi\left(\vec{r}\right) + \frac{1}{4 \pi} Q\:.
  \end{equation}
Integrate over all angles to obtain the zeroth moment equation
\begin{equation}
  \vec{\nabla} \cdot \vec{J} + \Sigma_a\phi  =  Q
  \label{eq:zeroth_moment_1g}
  \end{equation}
where $\vec{J} = \int_{4\pi} \vec{\Omega}\psi$ and $\Sigma_a$ is the absorption cross section.   Now consider the first moment equation:
  \begin{equation}
  \vec{\nabla} \cdot P + \Sigma_t J = 0
  \end{equation}
where $\vec{\nabla} \cdot P =  \int_{4\pi} \vec{\Omega} \vec{\Omega} \cdot \vec{\nabla} \psi$. It can be rewritten as: 

  \begin{equation}
  J= -\frac{1}{\Sigma_t} \vec{\nabla} \cdot P. 
  \end{equation}
  By adding and subtracting the diffusion coefficient, $D = \frac{1}{3\Sigma_t}$, times the gradient of the flux, it takes the form of a correction to Fick's Law. 
  \begin{equation}
  J = -D \vec{\nabla} \phi + D \vec{\nabla} \phi - \frac{1}{\Sigma_t} \vec{\nabla} P \\
  = -D \vec{\nabla} \phi - \vec{\textbf{D}} \phi
  \label{eq:fick_corr_1g}
  \end{equation}
  where 
 \begin{equation}
  \vec{\textbf{D}} (\psi) = \frac{\int_{4\pi} [\frac{1}{\Sigma_t} \vec{\Omega}_\mrm \vec{\Omega}_\mrm \cdot \vec{\nabla}\psi^{ho}_\mrm] - D \vec{\nabla} \phi^{ho}}{\phi^{ho}}.
  \label{eq:drift_vector}
  \end{equation} 
Where $\psi^{ho}$ indicates the solution of the higher order equation. Plugging \eqref{eq:fick_corr_1g} into \eqref{eq:zeroth_moment_1g} we have the following NDA equation:
  \begin{equation}
  \vec{\nabla}\cdot(-D \vec{\nabla} \phi - \vec{\bf D} \phi) + \Sigma_a \phi = Q \:. \label{eq:NDA_1g}
  \end{equation}
  
 
\subsection{Self-Adjoint Angular Flux}
In calculating the drift vector in the NDA equations, we use the scalar flux calculated by a ``higher-order equation." In this case, we will be using the Self-Adjoint Angular Flux Equation (SAAF) \cite{saaf} as our higher order equation. It is a reformulation of the transport equation that is more suitable for numerical linear solvers. 

The monoenergetic version of the Self-Adjoint Angular Flux equation appropriate for $S_n$ calculations is as follows:
\begin{equation}
    - \vec{\Omega} \cdot \vec{\nabla}\frac{1}{\Sigma_t}\vec{\Omega} \cdot \vec{\nabla} \psi + \Sigma_t \psi = \Sigma_s\phi + Q - \vec{\Omega} \cdot \vec{\nabla} \frac{(\Sigma_s\phi + Q)}{\Sigma_t}
    \label{eq:SAAF}
\end{equation}
Where $\vec{\Omega}$ are the angles, $\Sigma_t$ is the total cross section, $\psi$ is the angular flux, $\Sigma_s$ is the scattering cross section, $\phi$ is the scalar flux, and $Q$ is the fixed source.

\subsection{Coupling NDA and SAAF}
In this work, we use SAAF as the higher order equation and NDA to accelerate the convergence of source iteration. The implementation of the algorithm is outlined below:

\begin{enumerate}
    \item Intitialize system, by setting $\vec{\textbf{D}}$ to 0 and solving \eqref{eq:NDA_1g} to get $\phi^0$ 
    \item Loop Until Convergence:
        \begin{enumerate}
            \item Solve \eqref{eq:SAAF} for $\psi^l$ using $\phi^{l-1}$ on RHS.
            \item Calculate drift vector, \eqref{eq:drift_vector}, using $\psi^l$
            \item Solve \eqref{eq:NDA_1g} for $\phi^l$
            \item Check $\phi^{l-1}, \phi^l$ for convergence
        \end{enumerate}
    \item Return $\phi$
\end{enumerate}

While we are able to replicate the results of \cite{Wang2013}, showing a significant reduction in the number of source iterations necessary when using NDA with SAAF as compared to SAAF alone, NDA only acclerates one layer of iteration. The presence of upscattering introduces another layer of iteration: Gauss-Seidel iteration in energy. In the following section we derive an acceleration scheme for the outer layer of iteration.


\section{Methods of Discretization}
The angular flux, $\psi$, is a function of space, angle, and energy. In the solution process, each one of those dimensions is discretized. There are several choices that have to be made regarding discretizations. In this work, we endeavor to show equation forms that are discretization agnostic as well as showing formulations unique to the particular discretization methods we chose to implement. 

\subsection{Angular Discretization}

Angular discretization on the left hand side of \eqref{eq:transport} is handled via the discrete ordinates ($S_N$) method, a finite-element collocation method \cite{Lathrop1965}. We assume the sources are isotropic and do not perform any additional expansion, however they can be expanded via Spherical Harmonics and the methods will still hold. 

The $S_N$ method evaluates the equation at number of discrete angles or ``ordinates" and then sums over all angles with their given weights to perform a quadrature. To choose quadrature points, an octant of the unit sphere is discretized into several levels. At each level several nodes are chosen.We use a Gauss-Chebyshev angular quadrature set, which can be thought of as a product set, combining a one dimensional Gaussian Quadrature along the polar angles and an equally-weighted Chebyshev quadrature along the azimuthal angles \cite{jarrel-thesis}.

\begin{figure}[H]
    \centering
    \includegraphics[width=.5\textwidth]{fig/SNPoints.png}
    \caption{Equally-Weighted Gauss-Chebyshev Quadrature Points \cite{Lathrop1965}}
    \label{fig:SN}
\end{figure}


The discretized, steady-state, $S_N$ transport equation is given as follows,

 \begin{equation}
  \vec{\Omega}_\mrm \cdot \vec{\nabla} \psi_\mrm \left(\vec{r}\right)+ \Sigma_{\mm{t}}\left(\vec{r}\right)\psi_\mrm = \frac{1}{4 \pi} \Sigma_{\mm{s}}\left(\vec{r}\right) \phi\left(\vec{r}\right) + \frac{1}{4 \pi} Q
  \label{eq:transport-angular}
 \end{equation}
where $\mrm$ is the angular index and $\phi = \sum\limits_{\mrm=1}^M \omega_\mrm \psi_m$.

\subsection{Energy Discretization}
In our treatment of energy, we divide the full energy spectrum into several energy groups. By convention, the highest energy group is given the index, 1, with the index number going up until it reaches the lowest energy group. In expanding to multiple energy groups, we must take into account scattering from one group, $\rg'$ to another, $\rg$, denoted as $\rg' \rightarrow \rg $. 

The energy discretized, steady-state, transport equation is

 \begin{equation}
  \vec{\Omega} \cdot \vec{\nabla} \psi_\rg \left(\vec{r}\right)+ \Sigma_{\mm{t}, \rg}\left(\vec{r}\right)\psi_\rg = \frac{1}{4 \pi} \sum\limits_{\rg'=1}^{G}\Sigma_{\mm{s}, \rg' \rightarrow \rg}\left(\vec{r}\right) \phi_\rg\left(\vec{r}\right) + \frac{1}{4 \pi} Q_\rg.
  \label{eq:transport-energy}
 \end{equation}


\subsection{Spatial Discretization}
In this work, we choose to discretize in two dimensions, assuming uniformity in the third, however all formulations could be extended to be truly 3D. Spatial discretization methods for the transport equation are usually performed using commonly known differential equation discretization techniques such as the finite difference, finite volume, or finite element methods. In this work we discretize using the finite element method on triangular elements (described in detail in Appendix \ref{sec:spatial}), however TG-NDA can be used with any spatial discretization. 






\chapter{Two-Grid, Nonlinear Diffusion Acceleration}
\label{sec:derivation}

\section{Derivation of TG-NDA}
A condensed version of this derivation first appeared in our previous work \cite{Ramirez2017}. In this chapter we rederive the formulation with more detailed explanation.

Consider the first order, fixed-source, multigroup, steady state, \sn\ transport equation with isotropic scattering and internal source, where $\rg$ is the group index and $m$ is the angle index:
  \begin{equation}
  \vec{\Omega}_{\mrm,\rg}\cdot \nabla \psi_{\mrm,\rg} \left(\vec{r}\right)+ \Sigma_{\mm{t},\rg}\left(\vec{r}\right)\psi_{\mrm,\rg} = \frac{1}{4 \pi} \sum\limits_{\rg'=1}^\mm{G} \Sigma_{\mm{s},\rgp\to\rg}\left(\vec{r}\right) \phi_{\rg'}\left(\vec{r}\right) + \frac{1}{4 \pi} Q_\rg\:.
  \end{equation}
  Where $\phi_\rg = \sum\limits_{\mrm=1}^M w_\mrm \psi_{\mrm, \rg}$. Here, we assume vacuum boundary conditions. Using a similar technique as Peterson et al.\ used for the one group case, we multiply by angular weights and sum to get the zeroth moment equation \cite{morel-holo}.

  \begin{equation}
  \nabla \cdot J_\rg + \Sigma_{r,\rg}\phi_g  = \sum_{\substack{\rg'=1 \\ \rg' \neq \rg}}^G \Sigma_{s,\rg' \rightarrow \rg}\phi_{\rg'} + Q_\rg
  \label{eq:zeroth_moment}
  \end{equation}
where $J_\rg = \sum\limits_{\mrm=1}^{M}w_\mrm \vec{\Omega}\psi_{\mrm, \rg}$ and the removal cross section, $\Sigma_{r, \rg} = \Sigma_{t, \rg} - \Sigma_{s, \rg \rightarrow \rg}$ . 
  Now consider the first moment equation:
  \begin{equation}
  \nabla \cdot \overset{\text{\scriptsize$\leftrightarrow$}}{P_\rg} + \Sigma_{t, \rg} J_\rg = 0
  \end{equation}
where $\nabla \cdot \overset{\text{\scriptsize$\leftrightarrow$}}{P_\rg} =  \Omega_{m,\rg} \Omega_{m,\rg} \nabla \psi_{\rg}$. It can be rewritten as: 

  \begin{equation}
  J_\rg = -\frac{1}{\Sigma_{t, \rg}} \nabla \cdot \overset{\text{\scriptsize$\leftrightarrow$}}{P_\rg}. 
  \end{equation}
  By adding and subtracting the diffusion coefficient, $D_\rg = \frac{1}{3\Sigma_{t, \rg}}$, times the gradient of the flux, it takes the form of a correction to Fick's Law. 
  \begin{equation}
  J_\rg = -D_\rg \nabla \phi_\rg + D_\rg \nabla \phi_\rg - \frac{1}{\Sigma_{t, \rg}} \nabla \cdot \overset{\text{\scriptsize$\leftrightarrow$}}{P_\rg} \\
  = -D_\rg \nabla \phi_\rg - \vec{\textbf{D}}_\rg \phi_\rg
  \label{eq:fick_corr}
  \end{equation}
  where 
 \begin{equation}
  \vec{\textbf{D}}_\rg (\psi_\rg) = \frac{\sum\limits_{\mrm=1}^M w_\mrm [\frac{1}{\Sigma_t} \vec{\Omega}_\mrm \vec{\Omega}_\mrm \cdot \nabla\psi_{\mrm, \rg} - D_\rg \nabla \sum\limits_{\mrm=1}^M w_\mrm\psi_{\mrm, \rg}]}{\sum\limits_{\mrm=1}^M w_\mrm \psi_{\mrm, \rg}}.
  \end{equation} 
 Substituting \eqref{eq:fick_corr} in \eqref{eq:zeroth_moment} we have the following NDA equation:
  \begin{equation}
  \nabla\cdot(-D_\rg \nabla \phi_\rg - \vec{\bf D}_\rg \phi_\rg) + \Sigma_{r,\rg} \phi_\rg = \sum\limits_{\rgp\neq\rg}\Sigma_{\mm{s},\rg' \rightarrow \rg}\phi_{\rg'} + Q_\rg \:. \label{eq:NDA}
  \end{equation}
  The two grid method for upscatter acceleration involves two calculations at each iteration. We will refer to these intermediate calculations as half iterations. To add in the upscatter acceleration we must determine an equation for the error correction. We start with the $k + 1/2$ iteration, expressed as:
  \begin{equation}
  \begin{split}
  \nabla\cdot(-D_\rg \nabla \phi_\rg^{k+1/2} - \vec{\bf D}_\rg \phi_\rg^{k+1/2}) &+ \Sigma_{r,\rg} \phi_\rg^{k+1/2} =  \\ &\sum\limits_{\substack{\rg'=1}}^\mm{g-1} \Sigma_{\mm{s},\rg' \to\rg}\phi_{\rg}^{k+1/2} + \sum\limits_{\substack{\rg'=\rg+1}}^\mm{G} \Sigma_{\mm{s},\rg' \to\rg}\phi_{\rg'}^{k} + Q_\rg\:. 
  \end{split}
  \label{eq:k1/2}
  \end{equation}
  We next define two forms of error: The error of the scalar flux at each GS iteration, $\epsilon_\rg^{k+1/2}$, and the error in the scattering source from upscattering at each iteration, $R_\rg^{k+1/2}$.
  \begin{equation}
  \epsilon_\rg^{k+1/2} := \phi_\rg - \phi_\rg^{k + 1/2} \quad \mm{and} \quad R_\rg^{k+1/2} := \sum\limits_{\rg'=\rg+1}^\mm{G} \Sigma_{\mm{s},\rg' \rg}\left(\phi_{\rg'}^k - \phi_{\rg'}^{k + 1/2}\right)\:.
  \end{equation}
  $\epsilon_\rg^{k + 1/2}$ can be further broken into the spatial component $\phi_{\epsilon}$ and scattering spectrum $\xi_\rg$ as defined in \cite{morel-upscat,evans-upscat}.
  \begin{equation}
  \epsilon_\rg^{k+1/2} = \phi_{\epsilon}^{k+1/2}\left(\vec{r}\right)\xi_\rg, \hspace{5mm} \sum\limits_{\rg=1}^G \xi_\rg = 1.
  \end{equation}
  
  We calculate $\xi$ using the procedure outlined in \cite{evans-upscat}. Performing a Fourier analysis of the Gauss-Seidel iteration process, assuming isotropic scattering in an infinite medium, gives the following eigenproblem. 
  \begin{equation}
      (\textbf{T} - \textbf{S}_L - \textbf{S}_D)^{-1}\textbf{S}_U \xi = \rho(\lambda)\xi,
    \label{eq:eigenvector}
  \end{equation}
  where
  \begin{align*}
      \textbf{T} &= diag(\textbf{T}) =  \text{matrix of total cross sections by group} \\
      \textbf{S}_L &= lower(\textbf{S}) = \text{zeroth moments of the downscattering cross sections} \\
      \textbf{S}_D &= diag(\textbf{S}) = \text{zeroth moments of within group scattering cross sections} \\
      \textbf{S}_U &= upper(\textbf{S}) = \text{zeroth moments of the upscattering cross sections.}
  \end{align*}
  $\rho(\lambda)$ represents the eigenvalues of the system. $\xi$ is the eigenvector. The $\xi$ we are interested in is the eigenvector corresponding to the maximum eigenvalue of the system. $\xi_\rg$ is the $\rg$-th entry of that eigenvector. $\xi$ is material dependent and is recaluculated for each material in the problem.
  \par
  Subtracting Eq.\ \eqref{eq:k1/2}\ from Eq.\ \eqref{eq:NDA}\ and adding and subtracting $R_\rg^{k+1/2}$\ gives an exact formulation of the error, but for the sake of efficiency, we approximate the error using the following diffusion approximation:
  \begin{equation}
  \nabla\cdot(-D_\rg  \nabla \epsilon_\rg^{k+1/2} - \vec{\bf D}_\rg
  \epsilon_\rg^{k+1/2}) + \Sigma_{r,\rg}  \epsilon_\rg^{k+1/2} =  \sum\limits_{\substack{\rgp\neq\rg}} \Sigma_{\mm{s},\rg' \to\rg}  \epsilon_{\rg'}^{k+1/2} -  R^{k+1/2}_\rg \:.
  \end{equation}
  Using the eigenvector calculated in \eqref{eq:eigenvector}, we split the error into spatial and energy components and then integrate over all groups to collapse into a one group equation.
  \begin{equation}
  \nabla\cdot(-\left< D \xi \right> \nabla \phi_{\epsilon}^{k+1/2} - \left<\vec{\bf D} \xi
  \right> \phi_{\epsilon}^{k+1/2}) + \left< \Sigma_{a} \right> \phi_{\epsilon}^{k+1/2} = - \left< R^{k+1/2} \right> \:,\label{eq:collapsed}
  \end{equation}
  where $\left< X \right> = \sum\limits_{\rg = 1}^G X_\rg $ and $\Sigma_{a,\rg}  = \Sigma_{r,\rg}\xi_\rg - \sum\limits_{\rgp\neq\rg} \Sigma_{s,\rg'\rightarrow \rg} \xi_{\rg'}$.
  \par
  By solving \eqref{eq:collapsed} with the same spatial discretization as the NDA solve, we correct the scalar flux from the previous Gauss Seidel iteration as $\phi_\rg^{k+1} = \phi_\rg^{k+1/2} + \epsilon_\rg^{k+1/2}$.

\chapter{Iterative Solvers}
\label{sec:iterative}

\section{Iterative Methods for Neutron Transport}
Solving the transport equation involves several nested iterative solvers. To discuss these iterative methods it is useful to think of the transport equation in operator form
\begin{equation}
    \bf{L}\psi = \bf{S}\psi + \bf{Q} \:,
\end{equation}
where $\bf{L}$ is the matrix formed by discretizing the left-hand side, $[\hat{\Omega} \cdot \nabla + \Sigma(\vec{r}, E)]\psi(\vec{r}, \hat{\Omega}, E)$, using finite elements or any other discretization method and $\bf{S}\psi$ and $\bf{Q}$ are matrices representing the source terms. 

\subsubsection{Linear Solver}
The inner-most layer of iteration is the linear solve in which we treat the transport equation as a system of linear equations and solve for a vector, $\psi$, which represents the flux at several points in space. In order to solve the equation, it must take the form ${\bf A} x = b$ where $b$ is a constant vector. In our case, $b = {\bf S} \psi + {\bf Q}$, which is dependent on $\psi$ in the scattering source and makes $b$ not constant. To remedy this, we choose $\psi$ on the right hand side to be a constant. The method by which we choose this constant is detailed below. Now that $b$ is a constant vector, we are able to perform a linear solve. Solving $\textbf{A}x = b$ is well-studied problem and there are many iterative methods from which to choose. Most of these methods have already been implemented in forms optimized for performance in various linear algebra libraries such as LAPACK.

\subsubsection{``Inner" Iterations}
The inner iterations solve the space-angle component for each energy group. As was mentioned above, the scattering term on the right hand side of the transport equation (Eqn.~\ref{eq:transport}) depends on the angular flux, $\psi$. The inner iterations iterate on a guess of that angular flux so that it can be treated as a constant for the linear solves. The traditional method is called source iteration, which is explained in more detail below. A more advanced alternative to source iteration that is starting to gain popularity is a Krylov solver. 

\subsubsection{Multigroup or ``Outer Iterations"}
The multigroup iterations solve the energy component. When we have more than one energy group, we solve each group independently, performing source iteration on each one. In doing so, we must include the scattering from all other energy groups into the current group multiplied by the flux of the other groups. When there is no upscattering, meaning there is no scattering from a lower energy group to a higher energy group, we can solve each group sequentially, starting with the highest energy group, without any problems. No other groups scatter to group one, the highest energy group, the next group is only dependent on the flux from group one, which we just solved for, and so on. In the case of upscattering, an outer layer of iteration is added to converge the scattering source because lower energy groups \textit{do} contribute to higher energy groups. The most common multigroup solver is called the Gauss Seidel method. Other alternatives include Jacobi or multigroup Krylov. 

The three formulations that are discussed, NDA, TG-NDA, and SAAF, can be solved using any combination of the solvers listed above. We will explain in detail the solvers that were used in our implementation. 

\section{Implemented Solvers}
\subsection{Linear Solver}

For our linear solver we used SciPy's conjugate gradient (CG) solver \cite{SciPy} \cite{Shewchuck1994}. CG solves the system $Ax = b$ assuming $A$ is a real, symmetric, positive-definite matrix. By using the finite element method, we are guaranteed that our matrix $A$ satisfies those constraints, so CG is an applicable method for our problem.


\subsection{Within Group Solver - NDA}
The inner solves are handled by NDA, which is a correction to source iteration. In source iteration, the $\psi$ on the right hand side of the equation starts with an initial guess. At each iteration, the linear system is solved and $\psi$ is updated with the result of that solve. This continues until the values converge as can be seen in Algorithm \ref{ag:SI}.
\begin{algorithm}
\caption{Source Iteration}
\begin{algorithmic}
\While{$res > tol$} \Comment {Iteration Index $k$}
    \State \textbf{Set up} FEM discretization of Eq.
    \State ${\bf S} \psi \gets \sigma_s \phi^{k-1}$ \Comment {Assumes one group, see multigroup below.}
    \State \textbf{Solve} system $\textbf{L}\psi= {\bf S}\psi + {\bf Q}$
    \State $res \gets max(|\psi^{k} - \psi^{k-1}|)$
\EndWhile
\end{algorithmic}
\label{ag:SI}
\end{algorithm}


NDA adds an extra step to this process by first calculating the angular flux using a higher order equation and then solving the lower order equation using the result of the higher order equation to compute a correction. The result of the lower order solve is then used as the guess on the right hand side during the next iteration. \\
For ease of reference, we repeat the NDA+SAAF algorithm given in Ch. \ref{sec:intro} using iteration index $l$: 

\begin{enumerate}
    \item Intitialize system, by setting $\vec{\textbf{D}}$ to 0 and solving Eqn.~\eqref{eq:NDA_1g} to get $\phi^0$ 
    \item Loop Until Convergence:
        \begin{enumerate}
            \item Solve Eqn.~\eqref{eq:SAAF} for $\psi^l$ using $\phi^{l-1}$ on RHS.
            \item Calculate drift vector, Eqn.~\eqref{eq:drift_vector}, using $\psi^l$
            \item Solve Eqn.~\eqref{eq:NDA_1g} for $\phi^l$
            \item Check $\phi^{l-1}, \phi^l$ for convergence
        \end{enumerate}
    \item Return $\phi$
\end{enumerate}

\subsection{Multigroup Solver - Gauss Seidel}
When dealing with a multigroup problem, the scattering source is dependent not only on the flux from the current group, but from other groups as well. In the case that there is no upscattering, meaning that the flux from a lower energy group does not contribute to any higher energy group, we can solve sequentially, starting with the highest energy group and using the fluxes as we calculate them. In the case of upscattering, we must choose an initial guess for the fluxes of lower energy groups and then iterate until we find a convergent value. We use a method inspired by the Gauss Seidel method for linear equations. 
% So, what is the method? Is it GS? If not, what is it???
When iterating, we use the flux calculated in this iteration for the higher energy groups and the flux calculated in the previous iteration for the lower energy groups as can be seen in Algorithm \ref{ag:GS}.
\begin{algorithm}
\caption{Outer Iterations: Gauss Seidel}
\begin{algorithmic}
    \While {$res > tol$} \Comment {Iteration Index $l$}
        \For {$ g \in G$}
            \State \textbf{calculate} scattering source: \State $ {\bf S} \psi = \Sigma_{gg}\phi_g^{l+1} + \sum\limits_{g'=1}^{g-1} \Sigma_{gg'} \phi_{g'}^{l+1} + \sum\limits_{g'=g+1}^G \Sigma_{gg'}\phi_{g'}^l$
            \Procedure {Source Iteration on group $g$}{} 
        \EndProcedure
        \EndFor
        \State $res \gets max(|\phi^{l} - \phi^{l-1}|)$  \Comment {Check if sol. for each group has converged}
        \EndWhile
    \Return $\phi$
\end{algorithmic}
\label{ag:GS}
\end{algorithm}

\subsection{TG-NDA}
TG-NDA acclerates convergence by applying a correction at two different layers of iteration. NDA applies a correction using the drift vector at each source iteration. Two-Grid provides a correction at each Gauss Seidel iteration. The full scheme as implemented for this work is illustrated in Figure~\ref{fig:tgnda-graph} with the correction terms highlighted in blue. 

\begin{figure}[H]
    \centering
    \includegraphics[width=\textwidth]{fig/TGNDAchart.png}
    \caption{Full Iterative TG-NDA Scheme}
    \label{fig:tgnda-graph}
\end{figure}

\chapter{Numerical Results}
\label{sec:results}
\section{One Material}
For the first test problem we used a single material with significant upscattering throughout the domain with a constant source. 
\begin{center}
    \begin{tabular}{|c|c|c|}
    \hline
    & Runtime (s) & GS Iterations \\
    \hline
    TG-NDA & 2850.48293 & 7 \\
    NDA & 4352.18865 & 11 \\
    \hline
    \end{tabular}
\end{center}
\section{Two Materials}

\chapter{Conclusion \& Future Work}
\label{sec:future}

In this work I derived and implemented a two grid acceleration scheme for the nonlinear diffusion acceleration equations with neutron upscattering. In the test problems, a reduction in the number of Gauss-Seidel iterations necessary to resolve the upscattering is observed with roughly a factor of three improvement with no noticeable loss to accuracy. 

The current implementation is a proof of concept code developed in python. The Slaybaugh Lab intends to implement a performance optimized C++ version into the our neutron transport code, Bay Area Radiation Transport. Such an implementation would enable testing larger problems and aid in a better understanding of how the method scales.

\section{Possible Extensions to TG-NDA}
There are a number of modifications to my implementation of TG-NDA that could be interesting to explore. 


\subsection{Applications to Criticality Problems}
In this work I focused on fixed-source problems for shielding applications; however,
% this is the punctuation structure for howevers in the middle of sentences
 many materials commonly used in nuclear reactors, such as water, heavy water, or graphite, also exhibit significant upscattering. Performing criticality calculations using our method would be a straightforward extension involving only a layer of eigenvalue iteration to wrap around the existing implementation. 


\subsection{Compatibility with Other Methods}
In my implementation, I made choices regarding the discretization schemes, iterative methods, and equations I chose to use based on what I felt to be in common usage in the community. However, there are several other options to explore. In particular, 
\begin{enumerate}
\item $P_N$\ \textcolor{red}{and $S_N$-$P_N$}\ Angular Discretizations:
Our derivation is specific to the $S_N$ equations. It is possible to derive the higher order equation using $P_N$\ and the recently developed $S_N$-$P_N$\ hybrid scheme instead \cite{yaqi-wang-snpn,zheng-thesis,zheng-inl-report} and see if there is any effect on the convergence behavior of TG-NDA.
\item Discontinuous \textcolor{red}{and Other} Finite Elements:
In my implementation I use continuous finite elements to discretize in space. Discretizing NDA using discontinuous finite elements, as done in \cite{Schunert2017}, would change the form of TG-NDA and could change Gauss-Seidel convergence behavior. \textcolor{red}{Moreover, in a recent work \cite{zheng-ans17},\ Zheng et al\ developed a spatially hybrid finite elements which utilizes continuous finite elements in part of a problem and discontinuous finite elements anywhere else. It is of interest to develop a suitable NDA and TG-NDA for such hybrid finite elements}
\item Other Higher Order Equations:
I chose to pair NDA with SAAF, however other equations could be used as well, such as the even-parity equation \cite{Noh1996}\ \textcolor{red}{and least-squares transport equations and variants \cite{morel-holo,zheng-lspn,zheng-cdls}}.
\item Other Multigroup Solvers:
TG-NDA is specific to Gauss-Seidel, but a similar procedure could be used to create an acceleration scheme for other multigroup solvers such as Point Jacobi.
\end{enumerate}



\begin{appendices}
  \chapter{FEM on an Unstructured Triangular Grid}
  \label{sec:spatial}
  \section{Continuous FEM on an Unstructured Grid}
To discretize in space we will be using the continuous finite element method with bilinear basis functions on an unstructured triangular mesh. The mesh is generated with the Triangle library \cite{shewchuk96b} via a Delaunay triangulation algorithm. 
\par
To integrate over the triangles, we have implemented a 2nd degree Gaussian Quadrature approximation over the standard triangle,
\begin{align}
    \iint_{T_{st}} f(\xi, \eta) d\xi d\eta \approx \frac{1}{6}\left [ f\left( 0, \frac{1}{2} \right ) + f \left( \frac{1}{2}, 0 \right) + f \left( \frac{1}{2}, \frac{1}{2} \right)\right].
\end{align}
We are working with an unstructured grid of triangles, so we must additionally map nodes on the standard triangle to an arbitrary element. This is done by the \texttt{gaussNodes()} function in this \texttt{FEGrid} class. 
\begin{figure}[h]
    \centering
    \includegraphics[width=0.75\textwidth]{fig/TriangleMapping}
    \caption{Linear mapping between an element K and the standard triangle}
    \label{fig:my_label}
\end{figure}
The mapping is given as follows
\begin{align}
    x &= P(\xi, \eta) = x_1 + \xi(x_2 - x_1) + \eta(x_3 - x_1) \\
    y &= Q(\xi, \eta) = y_1 + \xi(y_2 - y_1) + \eta(y_3 - y_1)
\end{align}
After applying the transformation we have,
\begin{align}
 \iint_{K} F(x, y) dx dy = \iint_{T_{st}}F(P(\xi, \eta), Q(\xi, \eta))|J(\xi, \eta)|d\xi d\eta
\end{align}
where $|J(\xi, \eta)|$ is the Jacobian of the transform which is equal to $2Area_K$. This gives the final quadrature rule as
\begin{align}
 \iint_{K} &F(x, y) dx dy = 2A_k\iint_{T_{st}}F(P(\xi, \eta), Q(\xi, \eta))d\xi d\eta \\
 &\approx \frac{1}{6}\left [ F\left( P(0), Q(\frac{1}{2}) \right) + F \left( P(\frac{1}{2}), Q(0) \right) + F \left( P(\frac{1}{2}), Q(\frac{1}{2}) \right)\right].
\end{align}

\section{Weak Form of NDA}
Similar to the higher order equation, we discretize  Eq. 4 via the continuous finite element method by multiplying by a test function and integrating over the domain. This gives the following weak form:

Find $\varphi_{d, g} \in W_D$ such that

  \begin{align}
  \left(D_\rg \nabla \varphi_\rg^{k+1/2}, \nabla \varphi^*_\rg\right)_\mathcal{D} &+ \left(\vec{\bf{D}_\rg}\varphi_\rg^{k+1/2} , \nabla \varphi^*_\rg\right)_\mathcal{D} +  \left(\sigma_{r,\rg} \varphi_{\rg}^{k+1/2}, \varphi^*_\rg\right)_\mathcal{D} = \nonumber \\
   \left(\sum\limits_{\substack{\rg'=1}}^\mm{g-1} \sigma_{\mm{s},\rg' \to\rg}\varphi_{\rg}^{k+1/2}, \varphi^*_\rg\right)_\mathcal{D} &+ \left(\sum\limits_{\substack{\rg'=\rg+1}}^\mm{G} \sigma_{\mm{s},\rg' \to\rg}\varphi_{\rg}^{k}, \varphi^*_\rg\right)_\mathcal{D} 
  + \left(S_\mathrm{f,g}, \varphi^*_\rg\right)_\mathcal{D}  \label{k1/2}
  \end{align}


The error (Eq. 8) is similarly discretized via CFEM.
  \begin{align}
  \left(\left< D\xi \right >\nabla \varphi_{\epsilon}^{k+1}, \nabla \varphi^* \right)_\mathcal{D} + \left(\left<\vec{\bf{D}\xi} \right>\varphi_{\epsilon}^{k+1}, \nabla \varphi^* \right)_\mathcal{D} &+ \left(\left<\sigma_r \right>\phi_{\epsilon}^{k+1}, \varphi^* \right)_\mathcal{D}  \nonumber \\= \left(\left<R^{k+1} \right>, \varphi^* \right)_\mathcal{D} &+ \left(\left<S_\mathrm{f,g} \right>, \varphi^* \right)_\mathcal{D} 
  \end{align}

\section{Weak Form of the Higher Order Equation}
We apply a finite element discretization to the higher order equation, SAAF, by first multiplying Eq. \ref{eq:SAAF} by a test function $\psi*$ and integrating over the domain $D$.

\begin{equation}
    \left ( -\vec{\Omega} \cdot \vec{\nabla}\frac{1}{\sigma_t}\vec{\Omega} \cdot \vec{\nabla} \psi, \psi* \right )_D + \left ( \sigma_t \psi, \psi* \right )_D = \left ( Q, \psi* \right)_D - \left ( \vec{\Omega} \cdot \vec{\nabla}\frac{Q}{\sigma_t}, \psi* \right)_D
\end{equation}
Integrating by parts,

\begin{align*}
        \left ( \vec{\Omega}\frac{1}{\sigma_t}\vec{\Omega}\cdot \vec{\nabla}\psi, \vec{\nabla}\psi* \right)_D &-     \left ( \vec{\Omega}\cdot \hat{n} \frac{1}{\sigma_t}\vec{\Omega} \cdot \vec{\nabla} \psi,\psi* \right)_{\Gamma} + \left ( \sigma_t \psi, \psi* \right )_D = \\
        \left ( Q, \psi* \right)_D &+ \left ( \vec{\Omega} \frac{Q}{\sigma_t}, \vec{\nabla}\psi* \right)_D - \left ( \vec{\Omega}\cdot \hat{n} \frac{Q}{\sigma_t}, \psi* \right)_{\Gamma} 
\end{align*}


\end{appendices}

\bibliography{mybibfile.bib}
\bibliographystyle{plain}
\end{document}
